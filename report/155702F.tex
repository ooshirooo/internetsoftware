\documentclass[11pt,a4paper]{jsarticle}

\usepackage[dvipdfmx]{graphicx}
\usepackage{fancyvrb}
\usepackage{ascmac}
\usepackage{mediabb}

%\setlength{\textwidth}{179mm}
\setlength{\textheight}{250mm}
\setlength{\topmargin}{-2cm}
%\setlength{\oddsidemargin}{-1cm}
%\setlength{\evensidemargin}{-1cm}

\begin{document}

\title{インターネットソフトウェア report}
\author{155702F 大城由也}
\date{2017/8/24}

\maketitle
\newpage
\section*{講義の内容}
\subsection*{インターネットシステムの基礎知識}
主な内容:インターネットの変遷や、そこに関する装置・技術の概要の解説。
\begin{itemize}
\item OSI7階層モデル
\item インターネットはのパケット交換はIPルーティングとスイッチングによって実現している。
\item IPルーティングを実現しているのはルーター。
\item スイッチングを実現しているのはスイッチングハブ。
\item サーバーシステムはOSとサーバーアプリケーションで構成。
\item OSは基本機能を提供。高次元のサービスは提供しない。
\item アプリケーションはOSの基本機能を利用して高次元のサービスを提供。
\item セキュリティの実装法はネットワークシステムベースとサーバーシステムベースの2種類ある。
\item ネットワークシステムベースのセキュリティは、「攻撃が来る前に防ぐ」方法であり、アクセスフィルタやUTMの導入などが具体例である。
\item サーバーシステムベースのセキュリティは、「攻撃が来た時に防ぐ」方法であり、セキュリティ対策ソフトの導入やアクセス制御などが具体例である。
\end{itemize}


\subsection*{TCP/IPプログラミング演習(C言語)}
主な内容:ソケットを用いたパケット通信をプログラミングで実装。\\
 取り組んだ課題は「SMTPクライアントの作成」。メールサーバーを使用し、自身のアドレスや用意されたアドレスにメールを送信するプログラムの作成を行った。\\
 リンク先のGitリポジトリに作成したプログラムのソースコードを掲載した。\\
 https://github.com/ooshirooo/internetsoftware/blob/master/821/smtp.c


\subsection*{スイッチネットワーク構築演習}
情報工学実験IIで学習済みのため欠席


\subsection*{セキュリティ演習(OS\&サーバー編)}
主な内容:VMを構築し、セキュリティ設定を行った。\\
 行った内容は以下の5つ。
\begin{itemize}
\item ポート番号を確認して、不要なサービスに対し停止措置・ファイアウォールの設定を行う。
\item 一般ユーザーへのroot権限での実行の許可設定。
\item rootユーザー、パスワードなしユーザーのアクセス制限。
\item 鍵認証の設定、鍵認証のみでアクセスできるような制限の設定。
\item アクセス可能ユーザーの制限。
\end{itemize}


\newpage
\subsection*{セキュリティ演習(ツール編)}
主な内容:セキュリティチェックツール(Nmap, tcpdump, Wireshark)を使用したポートスキャン、パケットキャプチャを学んだ。\\
 Nmapはポートスキャンを行うツールであり、対象のOSやサーバーアプリケーションの種類、バージョンを調査可能。これを使用すれば、使用しているディストリビューションの状態が明らかになるのと同時に脆弱性も判明する。\\
 tcpdumpはパケットキャプチャを行うツールであり、パケットキャプチャを行うことで他人の通信パケットから情報の奪取が行える。奪取された情報はなりすましなどをする際に悪用される。\\
 Wiresharkはパケットキャプチャ機能に加え、プロトコル解析機能も有する。この機能により、実際にどのような内容の通信が行われていたのかをより具体的に知ることができる。


\section*{講義で得られたこと、今後取り上げて欲しい内容、感想}
今回の講義では、VMの構築方法(OSで行った時は理解できていない部分が多々あった)と、構築したVMに対するセキュリティ設定、セキュリティチェックを行う際使用するツールの使い方を新たに学んだ。もう一度自身で一からVMを構築してみたいと興味が湧いたので、さらに学習していきたい。\\
 時間があれば他のポートで展開されているサービスに関しても学習してみたかった(特にDNS)。\\
 教わる内容が多かったので、得るものも多かった。特に、インターネットの分野に興味が湧いたのは大きい収穫だと思う。将来就きたい職種はインターネットとは切り離せないので、モチベーションがあるうちに色々学ぼうと思う。

\end{document}